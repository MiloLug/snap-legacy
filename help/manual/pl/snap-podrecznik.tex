% !TeX spellcheck = pl

\documentclass{report}

\usepackage{baskervald}  % Default font
\usepackage{cabin}  % Sans serif font
\usepackage[T1]{fontenc}
\usepackage[utf8]{inputenc}
\usepackage{setspace} \onehalfspacing
\usepackage{graphicx}
\usepackage{textcomp}
\usepackage{hyperref}

% This macro produces a "Snap!" logo.
%
% Note that in Polish (and other languages), nouns are inflected. The form "Snap!" with suffix looks plain stupid, so the macro takes a suffix as argument. If the suffix is non-empty, no exclamation mark is generated, and the suffix is used instead.
\newcommand{\Snap}[1]{%
	\textsf{%
		Snap%
		\ifx\\#1\\\textit{!\@}%
		\else #1%
		\fi%
	}%
}

\newcommand{\code}[1]{\textsf{#1}}
\newcommand{\defaultGraphicsScale}{0.6}

\renewcommand{\thechapter}{\Roman{chapter}}
\renewcommand{\thesection}{\Alph{section}}
\renewcommand{\thesubsection}{}

\newcommand{\inlinepic}[1]{%
	\raisebox{-4pt}{%
		\includegraphics[scale=\defaultGraphicsScale]{#1}%
	}%
}

\newcommand{\bigpic}[1]{
	\begin{center}
	\includegraphics[scale=\defaultGraphicsScale]{#1}
	\end{center}
}

\usepackage[polish]{babel}
\usepackage{polski}
\frenchspacing
\usepackage{indentfirst}

\begin{document}

\title{\Snap{} \\ Podręcznik użytkownika}
\author{Brian Harvey \and Jens M\"{o}nig}
\date{}

\maketitle

\tableofcontents

\chapter*{}
\section*{Podziękowania}

Mieliśmy ogromne szczęście do mentorów. Jens zdobył dużo doświadczenia pracując wśród pionierów Smalltalka: Alana Kaya, Dana Ingallsa i~reszty ekipy, która wynalazła komputery osobiste i~programowanie obiektowe w~najlepszych dniach firmy Xerox PARC. Pracował z~Johnem Maloneyem z~zespołu Scratcha w~MIT\footnote{Massachusetts Institute of Technology, amerykańska uczelnia techniczna --- przyp. tłum.}, autorem platformy graficznej Morphic, wciąż stanowiącej fundament \Snap{a}. Znakomity projekt języka Scratch, autorstwa Lifelong Kindergarten Group z~MIT Media Lab, odgrywa w~\Snap{ie} kluczową rolę.

\textbf{\emph{Nasza poprzednia wersja, BYOB, była bezpośrednią modyfikacją kodu źródłowego Scratcha. \Snap{} został napisany od zera, lecz struktura jego kodu oraz interfejs użytkownika pozostają mocno zakorzenione w~Scratchu. Z~kolei zespół Scratcha, który mógłby widzieć w~nas rywali, przyjął nas ciepło i~okazał nam całkowite wsparcie.}}

Brian zdobywał szlify w~MIT oraz Stanford Artificial Intelligence Labs\footnote{Laboratorium sztucznej inteligencji na Uniwersytecie Stanforda --- przyp. tłum.}, gdzie uczył się pod okiem Johna McCarthy'ego, twórcy Lispa, oraz Geralda~J. Suss\-mana i~Guya Steele'a, twórców języka Scheme. Zdobywał również wiedzę od wielu innych wybitnych informatyków, w~tym autorów najlepszej książki z zakresu informatyki --- \emph{Struktury i~interpretacji programów komputerowych}: Hala Abelsona, Geralda~J. Suss\-mana i~Julie Suss\-man.

\textbf{\emph{Za starych dobrych czasów mawialiśmy w~MIT Logo Lab: ,,Język Logo to Lisp w przebraniu BASIC-a''. Dziś, ze swoimi pierwszoklasowymi procedurami, zasięgami leksykalnymi~i pierwszoklasowymi kontynuacjami, \Snap{} jest jak Scheme w~przebraniu Scratcha.}}

Szczęśliwym zrządzeniem losu, poprzez forum Scratch Advanced Topics, poznaliśmy wspaniałą grupę błyskotliwych uczniów gimnazjów~(!\@) i liceów. Kilku z nich wniosło swój wkład w~kod \Snap{a}: Kartik Chandra, Nathan Dinsmore, Connor Hudson i~Ian Reynolds. Ponadto wielu zgłosiło pomysły i~raporty błędów podczas testowania wersji alfa. Wśród studentów Uniwersytetu Kalifornijskiego w~Berkeley, którzy przyczynili się do rozwoju kodu, znajdują się Michael Ball, Achal Dave, Kyle Hotchkiss, Ivan Motyashov i~Yuan Yuan. Wymienianie wszystkich tłumaczy zajęłoby zbyt wiele miejsca, ale można ich odnaleźć w~okienku ,,O Snap!...'' dostępnym w~programie. Niniejsze dzieło powstało częściowo w~ramach grantu nr~1143566 udzielonego przez National Science Foundation, a częściowo przy wsparciu firmy MioSoft.

\clearpage

\begin{center}
\bf \Huge \Snap{} \\
Podręcznik użytkownika \\
\huge Wersja 4.0 \vspace{40pt}
\end{center}

\Snap{} to rozszerzona reimplementacja języka Scratch (\url{http://scratch.mit.edu}), która pozwala na tworzenie własnych bloków (ang.\ \textit{Build Your Own Blocks}; stąd dawna nazwa \Snap{a} --- BYOB). Opisywany tu język obsługuje pierwszoklasowe listy, procedury i~kontynuacje. Te dodatkowe możliwości sprawiają, że nadaje się on do przeprowadzenia poważnego wstępu do informatyki dla uczniów liceów i szkół wyższych. Aby uruchomić środowisko \Snap{}, wystarczy otworzyć przeglądarkę internetową i~wpisać adres \url{http://snap.berkeley.edu/run}, aby zacząć pracę z~minimalnym zestawem bloków. Można też użyć adresu \url{http://snap.berkeley.edu/init}, aby załadować niewielki zestaw dodatkowych bloków. Wiąże się to z~nieco wolniejszym ładowaniem, ale jest zalecane dla wygody użytkowników (w~dalszej części podręcznika będziemy zakładali korzystanie z~tej właśnie metody).

\clearpage

\chapter{Bloki, skrypty i duszki}

W~tym rozdziale poznamy kilka cech języka \Snap{} odziedziczonych po Scratchu; doświadczeni użytkownicy Scratcha mogą przejść od razu do sekcji~\ref{sec:zagnieżdżanie-duszków}.

\Snap{} jest językiem programowania --- notacją, przy pomocy której możemy powiedzieć komputerowi, co ma zrobić. Jednak w~odróżnieniu od większości innych, \Snap{} jest językiem wizualnym; programując w~nim, zamiast posługiwać się klawiaturą, używamy metody ,,przeciągnij i~upuść'', dobrze znanej użytkownikom komputerów.

Uruchom teraz środowisko \Snap{}. Powinieneś zobaczyć ekran podzielony na kilka obszarów:

\begin{center}
\includegraphics[width=\textwidth]{window-regions}
\end{center}

(Proporcje tych stref mogą się różnić, w~zależności od rozmiaru i~kształtu okna przeglądarki.)

Program w~języku \Snap{} składa się z~jednego lub więcej \emph{skryptów}, te zaś z~kolei --- z~\emph{bloków}. Oto przykładowy skrypt:

\label{fig:typical-script}
\bigpic{typical-script}

Na powyższy skrypt składa się pięć bloków w~trzech różnych kolorach, odpowiadających trzem z~ośmiu \emph{palet} z~blokami. Obszar palet, znajdujący się po lewej stronie okna, pokazuje jedną paletę na raz. Do zmiany widocznej palety służy osiem przycisków znajdujących się tuż nad tym obszarem. Bloki ciemnożółte, widoczne w~naszym skrypcie, pochodzą z~palety Kontrola; zielone z~palety Pisak, a~niebieskie --- z~palety Ruch. Aby złożyć taki skrypt, należy poprzeciągać odpowiednie bloki z~palet do \emph{obszaru skryptu}, umiejscowionego na środku okna. Kiedy układamy jeden blok pod drugim w~taki sposób, aby wcięcie dolnego bloku znalazło się w~pobliżu wypustki tego powyżej, bloki łączą się ze sobą (ang. \textit{snap together}; stąd nazwa języka \Snap{}):

\bigpic{snapping-blocks}

Pozioma biała linia sygnalizuje, że jeśli puścimy zielony blok, połączy się on z~wypustką ciemnożółtego.

\subsection{Bloki-czapki i bloki poleceń}

Na górze skryptu znajduje się \emph{blok-czapka}, który określa, kiedy skrypt ma zostać wykonany. Nazwy bloków-czapek zazwyczaj zaczynają się słowem ,,\code{kiedy}''; nasz przykładowy skrypt powinien zostać uruchomiony w~momencie kliknięcia zielonej flagi, znajdującej się w pobliżu prawej strony paska narzędzi \Snap{a}. (Pasek ten jest częścią okna programu \Snap{}; nie chodzi tutaj o pasek menu przeglądarki lub systemu operacyjnego.) Skrypt nie musi posiadać czapki, jednak w~takim przypadku zostanie wykonany tylko wtedy, gdy użytkownik sam go kliknie. Skrypt nie może mieć więcej niż jednej czapki; jej charakterystyczny kształt służy łatwiejszemu zapamiętaniu tej szczególnej własności.

Pozostałe bloki w naszym skrypcie to \emph{bloki poleceń}. Każdy z~nich oznacza jakąś akcję, którą \Snap{} potrafi wykonać. Na przykład blok \inlinepic{move-10-steps} nakazuje duszkowi\footnote{W grafice komputerowej słowem ,,duszek'' (ang. \textit{sprite}) nazywa się ruchomy obiekt na ekranie --- przyp. tłum.}, czyli strzałce na \emph{scenie} po prawej stronie okna, aby przesunął się o~dziesięć kroków do przodu w~kierunku, w~którym jest zwrócony. Każdy krok to niewielka odległość na ekranie. Wkrótce przekonamy się, że na scenie może być więcej duszków, a~każdy z nich może mieć własne skrypty. Ponadto duszki nie muszą wyglądać jak strzałki; ich kostiumy mogą być dowolnymi obrazkami. Kształt bloku \code{przesuń} ma za zadanie przypominać klocek, skrypt zaś jest jak wieża z klocków. Słowa ,,blok'' będziemy używać dla oznaczenia zarówno graficznego symbolu na ekranie, jak i~procedury (akcji) jaką ten blok wykonuje.

Liczbę 10 w powyższym bloku \code{przesuń} nazywamy jego \emph{parametrem}. Kliknąwszy na białym, zaokrąglonym polu, możemy wpisać w~jej miejsce dowolną inną. W przykładowym skrypcie ze strony \pageref{fig:typical-script} parametrem jest liczba 100. Jak się później okaże, pola parametrów mogą mieć kształty inne od zaokrąglonych; oznacza to wtedy, że akceptują one wartości inne niż liczby. Zobaczymy również, że zamiast wpisywać konkretne wartości w~pola, możemy nakazać komputerowi je obliczać. Ponadto blok może mieć więcej niż jeden parametr. Na przykład blok \code{leć}, znajdujący się mniej więcej w~połowie palety Ruch, przyjmuje trzy parametry.

Większość bloków poleceń ma kształt klocków, lecz niektóre, jak \code{powtórz} z~tego samego przykładu, wyglądają jak \emph{klamry}. Większość bloków klamrowych można znaleźć na palecie Kontrola. Wnętrze klamry jest szczególnym rodzajem pola parametru, który przyjmuje \emph{skrypt} jako parametr. W~przykładowym skrypcie blok \code{powtórz} ma dwa parametry: liczbę 4 oraz skrypt

\bigpic{typical-script-inner}



\section{Sprites and Parallelism}
\subsection{Costumes and Sounds}
\subsection{Inter-Sprite Communication with Broadcast}
\section{Zagnieżdżanie duszków: kotwice i części}
\label{sec:zagnieżdżanie-duszków}
\section{Reporter Blocks and Expressions}
\section{Predicates and Conditional Evaluation}
\section{Variables}
\subsection{Global Variables}
\subsection{Script Variables}
\section{Etcetera}
\chapter{Saving and Loading Projects and Media}
\section{Local Storage}
\subsection{Localstore}
\subsection{XML Export}
\section{Cloud Storage}
\section{Loading Saved Projects}
\chapter{Building a Block}
\section{Simple Blocks}
\subsection{Custom Blocks with Inputs}
\section{Recursion}
\section{Block Libraries}
\chapter{First Class Lists}
\section{The list Block}
\section{Lists of Lists}
\section{Functional and Imperative List Programming}
\section{Higher Order List Operations and Rings}
\chapter{Typed Inputs}
\section{Scratch's Type Notation}
\section{The \Snap{} Input Type Dialog}
\subsection{Procedure Types}
\subsection{Pulldown inputs}
\subsection{Input variants}
\subsection{Prototype Hints}
\subsection{Title Text and Symbols}
\chapter{Procedures as Data}
\section{Call and Run}
\subsection{Call/Run with inputs}
\subsection{Variables in Ring Slots}
\section{Writing Higher Order Procedures}
\subsection{Recursive Calls to Multiple-Input Blocks}
\section{Formal Parameters}
\section{Procedures as Data}
\section{Special Forms}
\subsection{Special Forms in Scratch}
\chapter{Object Oriented Programming}
\section{Local State with Script Variables}
\section{Messages and Dispatch Procedures}
\section{Inheritance via Delegation}
\section{An Implementation of Prototyping OOP}
\chapter{The Outside World}
\section{The World Wide Web}
\section{Hardware Devices}
\section{Date and Time}
\chapter{Continuations}
\section{Continuation Passing Style}
\section{Call/Run w/Continuation}
\subsection{Nonlocal exit}
\chapter{User Interface Elements}
\section{Tool Bar Features}
\subsection{The \Snap{} Logo Menu}
\subsection{The File Menu}
\subsection{The Cloud Menu}
\subsection{The Settings Menu}
\subsection{Stage Resizing Buttons}
\subsection{Project Control Buttons}
\section{The Palette Area}
\subsection{Context Menus for Palette Blocks}
\subsection{Context Menu for the Palette Background}
\section{The Scripting Area}
\subsection{Sprite Appearance and Behavior Controls}
\subsection{Scripting Area Tabs}
\subsection{Scripts and Blocks Within Scripts}
\subsection{Scripting Area Background Context Menu}
\subsection{Controls in the Costumes Tab}
\subsection{The Paint Editor}
\subsection{Controls in the Sounds Tab}
\section{Controls on the Stage}
\section{The Sprite Corral and Sprite Creation Buttons}

\end{document}

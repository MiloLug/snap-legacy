% !TeX spellcheck = pl

\documentclass{report}

\usepackage{baskervald}  % Default font
\usepackage{cabin}  % Sans serif font
\usepackage[T1]{fontenc}
\usepackage[utf8]{inputenc}
\usepackage{setspace} \onehalfspacing
\usepackage{graphicx}
\usepackage{textcomp}
\usepackage{hyperref}

% This macro produces a "Snap!" logo.
%
% Note that in Polish (and other languages), nouns are inflected. The form "Snap!" with suffix looks plain stupid, so the macro takes a suffix as argument. If the suffix is non-empty, no exclamation mark is generated, and the suffix is used instead.
\newcommand{\Snap}[1]{%
	\textsf{%
		Snap%
		\ifx\\#1\\\textit{!\@}%
		\else #1%
		\fi%
	}%
}

\newcommand{\code}[1]{\textsf{#1}}
\newcommand{\defaultGraphicsScale}{0.6}

\renewcommand{\thechapter}{\Roman{chapter}}
\renewcommand{\thesection}{\Alph{section}}
\renewcommand{\thesubsection}{}

\newcommand{\inlinepic}[1]{%
	\raisebox{-4pt}{%
		\includegraphics[scale=\defaultGraphicsScale]{#1}%
	}%
}

\newcommand{\bigpic}[1]{
	\begin{center}
	\includegraphics[scale=\defaultGraphicsScale]{#1}
	\end{center}
}

\usepackage[polish]{babel}
\usepackage{polski}
\frenchspacing
\usepackage{indentfirst}

\begin{document}

\title{\Snap{} \\ Podręcznik użytkownika}
\author{Brian Harvey \and Jens M\"{o}nig}
\date{}

\maketitle

\tableofcontents

\chapter*{}
\section*{Podziękowania}

Mieliśmy ogromne szczęście do mentorów. Jens zdobył dużo doświadczenia pracując wśród pionierów Smalltalka: Alana Kaya, Dana Ingallsa i~reszty ekipy, która wynalazła komputery osobiste i~programowanie obiektowe w~najlepszych dniach firmy Xerox PARC. Pracował z~Johnem Maloneyem z~zespołu Scratcha w~MIT\footnote{Massachusetts Institute of Technology, amerykańska uczelnia techniczna --- przyp. tłum.}, autorem platformy graficznej Morphic, wciąż stanowiącej fundament \Snap{a}. Znakomity projekt języka Scratch, autorstwa Lifelong Kindergarten Group z~MIT Media Lab, odgrywa w~\Snap{ie} kluczową rolę.

\textbf{\emph{Nasza poprzednia wersja, BYOB, była bezpośrednią modyfikacją kodu źródłowego Scratcha. \Snap{} został napisany od zera, lecz struktura jego kodu oraz interfejs użytkownika pozostają mocno zakorzenione w~Scratchu. Z~kolei zespół Scratcha, który mógłby widzieć w~nas rywali, przyjął nas ciepło i~okazał nam całkowite wsparcie.}}

Brian zdobywał szlify w~MIT oraz Stanford Artificial Intelligence Labs\footnote{Laboratorium sztucznej inteligencji na Uniwersytecie Stanforda --- przyp. tłum.}, gdzie uczył się pod okiem Johna McCarthy'ego, twórcy Lispa, oraz Geralda~J. Suss\-mana i~Guya Steele'a, twórców języka Scheme. Zdobywał również wiedzę od wielu innych wybitnych informatyków, w~tym autorów najlepszej książki z zakresu informatyki --- \emph{Struktury i~interpretacji programów komputerowych}: Hala Abelsona, Geralda~J. Suss\-mana i~Julie Suss\-man.

\textbf{\emph{Za starych dobrych czasów mawialiśmy w~MIT Logo Lab: ,,Język Logo to Lisp w przebraniu BASIC-a''. Dziś, ze swoimi pierwszoklasowymi procedurami, zasięgami leksykalnymi~i pierwszoklasowymi kontynuacjami, \Snap{} jest jak Scheme w~przebraniu Scratcha.}}

Szczęśliwym zrządzeniem losu, poprzez forum Scratch Advanced Topics, poznaliśmy wspaniałą grupę błyskotliwych uczniów gimnazjów~(!\@) i liceów. Kilku z nich wniosło swój wkład w~kod \Snap{a}: Kartik Chandra, Nathan Dinsmore, Connor Hudson i~Ian Reynolds. Ponadto, wielu zgłosiło pomysły i~raporty błędów podczas testowania wersji alfa. Wśród studentów Uniwersytetu Kalifornijskiego w~Berkeley, którzy przyczynili się do rozwoju kodu, znajdują się Michael Ball, Achal Dave, Kyle Hotchkiss, Ivan Motyashov i~Yuan Yuan. Wymienianie wszystkich tłumaczy zajęłoby zbyt wiele miejsca, ale można ich odnaleźć w~okienku ,,O Snap!...'' dostępnym w~programie. Niniejsze dzieło powstało częściowo w~ramach grantu nr~1143566 udzielonego przez National Science Foundation, a częściowo przy wsparciu firmy MioSoft.

\begin{titlepage}

\begin{center}
\bf \Huge \Snap{} \\
Podręcznik użytkownika \\
\huge Wersja 4.0 \vspace{40pt}
\end{center}

\Snap{} to rozszerzona reimplementacja języka Scratch (\url{http://scratch.mit.edu}), która pozwala na tworzenie własnych bloków (ang.\ \textit{Build Your Own Blocks}; stąd dawna nazwa \Snap{a} --- BYOB). Opisywany tu język obsługuje pierwszoklasowe listy, procedury i~kontynuacje. Te dodatkowe możliwości sprawiają, że nadaje się on do przeprowadzenia poważnego wstępu do informatyki dla uczniów liceów i szkół wyższych. Aby uruchomić środowisko \Snap{}, wystarczy otworzyć przeglądarkę internetową i~wpisać adres \url{http://snap.berkeley.edu/run}, aby zacząć pracę z~minimalnym zestawem bloków. Można też użyć adresu \url{http://snap.berkeley.edu/init}, aby załadować niewielki zestaw dodatkowych bloków. Wiąże się to z~nieco wolniejszym ładowaniem, ale jest zalecane dla wygody użytkowników (w~dalszej części podręcznika będziemy zakładali korzystanie z~tej właśnie metody).

\end{titlepage}

\chapter{Bloki, skrypty i duszki}

W~tym rozdziale poznamy kilka cech języka \Snap{} odziedziczonych po Scratchu; doświadczeni użytkownicy Scratcha powinni przejść od razu do sekcji~\ref{sec:zagnieżdżanie-duszków}.

This chapter describes the \Snap{} features inherited from Scratch; experienced Scratch users can skip to section B.

\Snap{} is a programming language---a notation in which you can tell a computer what you want it to do. Unlike most programming languages, though, \Snap{} is a visual language; instead of writing a program using the keyboard, the \Snap{} programmer uses the same drag-and-drop interface familiar to computer users.

Start \Snap{}. You should see the following arrangement of regions in the window:

\begin{center}
\includegraphics[width=\textwidth]{window-regions}
\end{center}

(The proportions of these areas may be different, depending on the size and shape of your browser window.)

A \Snap{} program consists of one or more \emph{scripts}, each of which is made of \emph{blocks}. Here's a typical script:

\bigpic{typical-script}

The five blocks that make up this script have three different colors, corresponding to three of the eight \emph{palettes} in which blocks can be found. The palette area at the left edge of the window shows one palette at a time, chosen with the eight buttons just above the palette area. In this script, the gold blocks are from the Control palette; the green block is from the Pen palette; and the blue blocks are from the Motion palette. A script is assembled by dragging blocks from a palette into the \emph{scripting area} in the middle part of the window. Blocks snap together (hence the name \Snap{} for the language) when you drag a block so that its indentation is near the tab of the one above it:

\bigpic{snapping-blocks}

The white horizontal line is a signal that if you let go of the green block it will snap into the tab of the gold one.

\subsection{Hat Blocks and Command Blocks}

At the top of the script is a \emph{hat} block, which indicates when the script should be carried out. Hat block names typically start with the word ``\code{when}''; in this example, the script should be run when the green flag near the right end of the \Snap{} tool bar is clicked. (The \Snap{} tool bar is part of the \Snap{} window, not the same as the browser's or operating system's menu bar.) A script isn't required to have a hat block, but if not, then the script will be run only if the user clicks on the script itself. A script can't have more than one hat block, and the hat block can be used only at the top of the script; its distinctive shape is meant to remind you of that.

The other blocks in this script are \emph{command} blocks. Each command block corresponds to an action that \Snap{} already knows how to carry out. For example, the block \inlinepic{move-10-steps} tells the sprite (the arrowhead shape on the \emph{stage} at the right end of the window) to move ten steps (a step is a very small unit of distance) in the direction in which the arrowhead is pointing. We'll see shortly that there can be more than one sprite, and that each sprite has its own scripts. Also, a sprite doesn't have to look like an arrowhead, but can have any picture as a costume. The shape of the \code{move} block is meant to remind you of a Lego\texttrademark{} brick; a script is a stack of blocks. (The word ``block'' denotes both the graphical shape on the screen and the procedure, the action, that the block carries out.)

The number 10 in the \code{move} block above is called an \emph{input} to the block. By clicking on the white oval, you can type any number in place of the 10. The sample script on the previous page uses 100 as the input value. We'll see later that inputs can have non-oval shapes that accept values other than numbers. We'll also see that you can compute input values, instead of typing a particular value into the oval. A block can have more than one input slot. For example, the \code{glide} block located about halfway down the Motion palette has three inputs.

Most command blocks have that brick shape, but some, like the \code{repeat} block in the sample script, are \emph{C-shaped}. Most C-shaped blocks are found in the Control palette. The slot inside the C shape is a special kind of input slot that accepts a \emph{script} as the input. In the sample script, the \code{repeat} block has two inputs: the number 4 and the script

\bigpic{typical-script-inner}



\section{Sprites and Parallelism}
\subsection{Costumes and Sounds}
\subsection{Inter-Sprite Communication with Broadcast}
\section{Zagnieżdżanie duszków: kotwice i części}
\label{sec:zagnieżdżanie-duszków}
\section{Reporter Blocks and Expressions}
\section{Predicates and Conditional Evaluation}
\section{Variables}
\subsection{Global Variables}
\subsection{Script Variables}
\section{Etcetera}
\chapter{Saving and Loading Projects and Media}
\section{Local Storage}
\subsection{Localstore}
\subsection{XML Export}
\section{Cloud Storage}
\section{Loading Saved Projects}
\chapter{Building a Block}
\section{Simple Blocks}
\subsection{Custom Blocks with Inputs}
\section{Recursion}
\section{Block Libraries}
\chapter{First Class Lists}
\section{The list Block}
\section{Lists of Lists}
\section{Functional and Imperative List Programming}
\section{Higher Order List Operations and Rings}
\chapter{Typed Inputs}
\section{Scratch's Type Notation}
\section{The \Snap{} Input Type Dialog}
\subsection{Procedure Types}
\subsection{Pulldown inputs}
\subsection{Input variants}
\subsection{Prototype Hints}
\subsection{Title Text and Symbols}
\chapter{Procedures as Data}
\section{Call and Run}
\subsection{Call/Run with inputs}
\subsection{Variables in Ring Slots}
\section{Writing Higher Order Procedures}
\subsection{Recursive Calls to Multiple-Input Blocks}
\section{Formal Parameters}
\section{Procedures as Data}
\section{Special Forms}
\subsection{Special Forms in Scratch}
\chapter{Object Oriented Programming}
\section{Local State with Script Variables}
\section{Messages and Dispatch Procedures}
\section{Inheritance via Delegation}
\section{An Implementation of Prototyping OOP}
\chapter{The Outside World}
\section{The World Wide Web}
\section{Hardware Devices}
\section{Date and Time}
\chapter{Continuations}
\section{Continuation Passing Style}
\section{Call/Run w/Continuation}
\subsection{Nonlocal exit}
\chapter{User Interface Elements}
\section{Tool Bar Features}
\subsection{The \Snap{} Logo Menu}
\subsection{The File Menu}
\subsection{The Cloud Menu}
\subsection{The Settings Menu}
\subsection{Stage Resizing Buttons}
\subsection{Project Control Buttons}
\section{The Palette Area}
\subsection{Context Menus for Palette Blocks}
\subsection{Context Menu for the Palette Background}
\section{The Scripting Area}
\subsection{Sprite Appearance and Behavior Controls}
\subsection{Scripting Area Tabs}
\subsection{Scripts and Blocks Within Scripts}
\subsection{Scripting Area Background Context Menu}
\subsection{Controls in the Costumes Tab}
\subsection{The Paint Editor}
\subsection{Controls in the Sounds Tab}
\section{Controls on the Stage}
\section{The Sprite Corral and Sprite Creation Buttons}

\end{document}

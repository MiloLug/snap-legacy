\usepackage[T1]{fontenc}
\usepackage[utf8]{inputenc}
\usepackage{baskervald}  % Default font
\usepackage[onehalfspacing]{setspace}
\usepackage{graphicx}
\usepackage{color}
\usepackage{textcomp}
\usepackage[hidelinks, pdfusetitle]{hyperref}
\usepackage{float}
\usepackage{ltxcmds}
\usepackage{fp}
\usepackage{tikz}
\usepackage{etoolbox}

\usetikzlibrary{calc}
\usetikzlibrary{fit}

% This macro produces a "Snap!" logo.
%
% Note that in Polish (and other languages), nouns are inflected. The form "Snap!" with suffix looks plain stupid, so the macro takes a suffix as argument. If the suffix is non-empty, no exclamation mark is generated, and the suffix is used instead.
\newcommand{\Snap}[1]{%
	\textsf{%
		Snap%
		\ifx\\#1\\\textit{!\@}%
		\else #1%
		\fi%
	}%
}

\newcommand{\code}[1]{\textsf{\textbf{#1}}}
\makeatletter
\newcommand{\defaultGraphicsScale}{0.6}
\makeatother

\renewcommand{\thechapter}{\Roman{chapter}}
\renewcommand{\thesection}{\Alph{section}}
\renewcommand{\thesubsection}{}

\makeatletter
\edef\defaultFontSize{\f@size}

% A dirty hack: Apparently, this is the only way to multiply a length by a constant factor in the code below. What a mess.
\newdimen\raiseLength

\newcommand{\inlinepicraised}[3]{%
	{%
		\FPeval\fontSizeRatio{\f@size / \defaultFontSize}%
		\FPeval\scale{#3 * \fontSizeRatio}%
		\raiseLength=#2%
		\raisebox{\fontSizeRatio\raiseLength}{%
			\includegraphics[scale=\scale]{#1}%
		}%
	}%
	% Add kerns if the next character is punctuation.
	\ltx@ifnextchar@nospace.{\,}{%
		\ltx@ifnextchar@nospace,{\,}{%
		}%
	}%
}
\makeatother

\newcommand{\inlinepic}[2][\defaultGraphicsScale]{%
	\inlinepicraised{#2}{-4pt}{#1}%
}

\newcommand{\inlinereporterpic}[2][\defaultGraphicsScale]{%
	\inlinepicraised{#2}{-1.5pt}{#1}%
}

\newcommand{\bigpic}[1]{
	\begin{figure}[H]
	\centering
	\includegraphics[scale=\defaultGraphicsScale]{#1}
	\end{figure}
}

\newcommand{\manualtitlepage}[2][]{ {
	\def\title{#2}
	\def\translator{#1}
	
	\begin{titlepage}

	\pgfdeclarelayer{background}
	\pgfsetlayers{background,main}
	\begin{tikzpicture}[remember picture, overlay, text depth=0]
	
	\definecolor{title blue}{HTML}{206C8F}
	\definecolor{title orange}{HTML}{F59201}
	\pgfdeclareimage[width=0.2\textwidth]{logo}{../common/logo}
	\pgfdeclareimage{cover picture}{../common/cover-picture}
	\coordinate (NW) at ($(current page.north west) + (0.5, -0.5)$);
	\coordinate (NE) at ($(current page.north east) + (-0.5, -0.5)$);
	\coordinate (SW) at ($(current page.south west) + (0.5, 0.5)$);
	\coordinate (SE) at ($(current page.south east) + (-0.5, 0.5)$);
	
	% Logo at the top left corner.
	\node[
		below right,
		align=left, 
		font=\Large,
		execute at begin node=\setlength\baselineskip{3ex}]
		at ($(NW) + (1.5, -1.5)$) {
		\pgfuseimage{logo}\\
		Build Your Own Blocks
	};
	
	% Version number.
	\node[
		below right,
		color=white,
		font=\fontsize{40pt}{48pt}\selectfont]
		at ($(NE) - (7, 5)$) {
		4.0
	};
	
	% Title. Note: the % at the end is necessary for correct spacing between the text and orange background.
	\node[
		left,
		text=white,
		align=right,
		inner sep=15pt,
		font=\fontsize{40pt}{42pt}\selectfont]
		(title text)
		at ($(NE) - (1.5, 9)$) {
		\title%
	};
	
	% The cover picture.
	\node[below left, inner sep=0] (cover picture)
		at ($(NE) - (0, 12)$) {
		\pgfuseimage{cover picture}
	};
	\draw[white, ultra thick]
		(cover picture.north west) -- (cover picture.north east);
	\draw[white, ultra thick]
		(cover picture.south west) -- (cover picture.south east);
	
	% Authors.
	\node[above right, color=white, align=left, font=\Large]
		at ($(SE) + (-7, 3)$) {
		Brian Harvey\\
		Jens M\"onig
	};
	
	% Translator.
	\ifdefempty{\translator}{}{
		\node [below right, color=white, align=left, font=\large]
			at ($(SE) + (-7, 2)$) {
			\translator
		};
	}
	
	\begin{pgfonlayer}{background}
		% Blue bar on the right.
		\fill[title blue] ($(NE) - (7.5, 0)$) rectangle (SE);
		
		% Title background stretches to the left until it reaches the left edge, which means (NW.x, title text.west.y).
		\coordinate (title background left point)
			at (NW |- title text.west);
		% Orange title background.
		\node[
			fit=(title text) (title background left point),
			shape=rectangle,
			fill=title orange,
			text=white,
			draw=white,
			ultra thick,
			align=right,
			inner sep=0,
			font=\fontsize{40pt}{42pt}\selectfont] {};
	\end{pgfonlayer}

	\end{tikzpicture}
	\end{titlepage}
} }